\section{Introduction}
Over the years, attacks on networks have become increasingly more
complex and covert and thus harder to detect. Post-mortem reports have 
shown network adversaries often spend upwards of 200 days inside the 
network before being discovered, and in some cases, more than 250 days
~\cite{attackers}. As long as the adversary continues to go undetected, 
these sophisticated network attacks allow an adversary to collect 
sensitive data and exfiltrate it from the network, resulting in severe 
financial damages.

To successfully carry out an attack on a network, the adversary might
typically perform a series of steps that goes like the following:
(1) get a foothold within the network, (2) using that foothold
identify both what the sensitive resources are and how to get to the
sensitive resources, (3) exfiltrate the data, and (4)
optionally have a plan to remove any evidence of their presence in the
network and end the attack, or simply continue to stay hidden in the 
network until they are detected. In this work, we analyze Step (2), also
known as \textit{network reconnaissance}, in deeper detail. More 
specifically, we look at an active probing mechanism for network
reconnaissance in a Software-Defined Network and defense mechanisms to 
detect such activities. Although network reconnaissance alone does not 
cause much harm to the network, it is often followed by an attack. Thus 
any attempt to perform network reconnaissance is a strong indication of 
a future attack on the network. We note simply preventing or detecting 
network reconnaissance alone is not enough for maintaining a secure
network, but our work can be integrated into a defense in depth 
approach to network security.

As mentioned in step 2 of the sample attack lifecycle, one of the goals 
of network reconnaissance is to identify both the sensitive resources 
and to access the sensitive resources before being able to exfiltrate 
data. Adversaries also try to gather information about the network 
itself, such as the types of security middleboxes or other network 
policies present. Identifying which hosts internal to the network contain
sensitive information may be a nontrivial task, but stealthily 
accessing the sensitive resources can be challenging. For example, the
foothold acquired in step 1  may not have access to the targeted 
resource directly or there is a network defense between the adversary
and the resource that will trigger an alarm of their presence. To
mitigate this defense, the adversary must identify vulnerable hosts 
safely reachable from their foothold and also contain a vulnerability
allowing them to move further into the network. These discovered 
vulnerable hosts act as \textit{stepping stones} to the targeted 
resource from the adversary's foothold and it may require multiple 
stepping stones to be compromised in order to access the targeted
resource. Once an adversary obtains a path to the targeted resources,
which may take multiple rounds of reconnaissance, they may begin
exfiltrating data from the network.
 
Network reconnaissance and methods to prevent and detect its activity 
has been an area of research since the late 1990's~\cite{phrack, 
de1999review}. Prior work in this domain has looked at 
network scanning attacks~\cite{de1999review, allman2007brief, 
xiaobing2001detection, barnett2008towards} that use spoofed IP packets
to gather intelligence in traditional networks. These active network 
reconnaissance attacks and the challenges in detecting the probes has 
also been well discussed in other previous 
works~\cite{shaikh2008network}. However, as Software Defined Networks 
are gaining popularity, we find it valuable to examine new network 
reconnaissance techniques introduced by their architecture. Network 
reconnaissance has yet to be reported as a security problem in SDN, but
in 2013 Song et al.~\cite{song2013network} discussed the possibility of 
network scanning attacks in SDN. This work is motivated to take 
proactive measures in identifying new security threats before SDN is 
deployed on a large scale. In addition, by the nature of the proposed 
stealthy reconnaissance techniques, it is possible these attacks are 
actively being used against SDN in the wild and there are no detection
mechanisms currently able to detect and report them as a problem. 

To illustrate the importance of identifying security problems during the
early stages of system deployment, consider the field of Operating System
Security. When operating systems were first developed, there were no 
concerns of security and it was assumed all users and programs would behave
as expected in a benign manner; however, as we all know, this quickly 
became an issue and systems needed to be re-engineered to provide much 
needed security features. As SDN is in its early phases, with the
potential for wide spread adoption, we have the opportunity to not only
tackle well known security issues seen in traditional networks but we
can also identify potential new attack vectors before they have the chance 
to cause any harm to deployed systems.

% List our contributions
In this work we analyze network reconnaissance in Software Defined
Networks and make the following key contributions:
\begin{enumerate}
  \item We present a new network reconnaissance attack in SDN that 
  gives an adversary, controlling a compromised switch, an enhanced view 
  of the entire network and enables information gathering in a way 
  infeasible in traditional networks.
  \item We present the design of an SDN application and implement a 
  prototype to detect the reconnaissance attack introduced above. Our
  prototype successfully detected reconnaissance probes with no false 
  positives or false negatives with an overhead around 3\% for both
  network throughput and latency.
  \item We setup a small scale but realistic lab environment to run 
  our experiments and present the results of our experiments.
\end{enumerate}
