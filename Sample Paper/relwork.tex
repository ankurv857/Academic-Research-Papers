\section{Related Work}
\label{sec:relwork}

Network Reconnaissance~\cite{shaikh2008network, barnett2008towards}
using various techniques has been identified as a threat to networks as
it empowers an adversary to gather crucial information about the network,
the devices present in the network, and the possible vulnerabilities that
can be used for further attacks or intrusions against the network. Prior 
work has mostly focused on looking at the different reconnaissance 
techniques used in traditional networks. Network scanning using 
\textit{Nmap}~\cite{lyon2009nmap} has been discussed as a threat to the
network~\cite{barnett2008towards, anbar2013investigating, allman2007brief}
as it allows an adversary to learn about open ports on and liveliness of
hosts in a network. Allman et al.~\cite{allman2007brief} studied the onset
of network scanning in the late 1990's and its evolution in terms of 
characteristics such as the number of scanners, targets, and probing 
patterns. Their work provided a preliminary examination of the network
scanning phenomenon. Anbar et al.~\cite{anbar2013investigating} presented
an investigative study on network scanning techniques and identified
many existing scanning methods. They also discussed how malicious code 
adopts scanning methods to find vulnerable hosts and services and explored
the current approaches to detect the presence of these scanning methods in
networks. Shaikh et al.~\cite{shaikh2008network} discussed different ways
of performing network reconnaissance using probes and presented the general
characteristics of such probes being used in the wild. In another work, 
Luckie et al.~\cite{luckie2008traceroute} used \textit{traceroute} under 
different IP protocols (TCP, UDP, ICMP,\ldots) and analyzed the routes 
taken when connecting to a set of routable IP addresses. In this work they
identified a list of known routers and well-known websites and explored 
how the paths taken by a probe varied with the choice of the protocol used
for a probe. Templeton et al.~\cite{templeton2003detecting} observed
adversaries tend to use spoofed packets for network reconnaissance and discussed
a wide variety of methods for detecting spoofed packets, including both
active and passive host-based methods as well as the more commonly
discussed routing-based methods. Xiaobing et
al.~\cite{xiaobing2001detection} also looked at this problem and
analyzed the existing scanning methods to draw the conclusion that the
then present detection and protection of scanning mainly aimed at
information concealment. They then presented a novel system of the
detection and protection named IEDP. 

Although prior work has focused on the dangers of network reconnaissance
on networks, they have been mostly targeted on the traditional networks.
With the increasing popularization of Software-Defined
Networks~\cite{feamster2013road, casado2007ethane}, it's important to
look at the similar problems in SDN and how it allows for previously
unseen ways of performing active network reconnaissance, which were
infeasible in traditional networks. There have been a number of
proposals using SDN to enhance network security. 
Avant-Guard~\cite{Shin13Avant} specifically focuses on addressing the
communication bottleneck between the control and data planes by
identifying malicious traffic, including network scanning and
denial-of-service. Song et al.~\cite{song2013network} were the first to
look at scanning attacks in SDN and proposed a solution by reactively
hiding critical resources in response to such attacks. The authors
implemented a security service that responds to network scanning
predefined policies by redirecting attackers to a honeynet and confuse
attackers by providing fake scanning results.

The idea of network devices, such as switches, being compromised and
used for attacks against a Software-Defined Network was discussed in
2015~\cite{dpm+15, hxw+15}. Hong et al.~\cite{hxw+15} studied the
popular and major SDN controllers and found that a lot of them are
subject to the Network Topology Poisoning Attacks. They investigated
mitigation methods against Network Topology Poisoning Attacks and
presented \textit{TopoGuard}, which is a security extension to SDN
controllers and provides automatic and real-time detection of Network
Topology Poisoning Attacks. Dhawan et al.~\cite{dpm+15} focused on
several attacks targeting  SDN controllers that violate network topology
and data plane forwarding, and can be mounted by compromised network
entities, such as end hosts and soft switches. They presented
\textit{SPHINX}, which uses flow graphs and a custom policy language, to
detect both known and potentially unknown attacks on network topology
and data plane forwarding originatgging within an SDN environment.

Although these previous works have focused on detecting spoofed packets from
physical hosts and have looked into link spoofing attacks by compromised switches,
they have not addressed the problem of the compromised switches generating spoofed
packet\_in messages. Since packets from a host are only checked on the switch they
are directly connected to and not at each hop in the network, previous
approaches are unable to detect packets generated at a switch. Making
detection even more difficult, the probes packets in our attack are not
propagated to other switches or hosts in the network thus detection lies
solely on identifying the probe packets as they are first made. Our work 
extends the ideas of network reconnaissance in SDN and looks at it
from a different perspective. We identify how an adversary can leverage
the architecture of SDN to launch network probes from a compromised 
switch in ways not seen in traditional networks and undetectable by
previous SDN security application. We then present a way to detect these
reconnaissance attacks and implement the design for an SDN app that can 
perform this detection.
