\section{Evaluation}
\label{sec:eval}

\begin{table}[t]
\caption{Network Performance Overhead}
\vspace{0.1 in}
\centering
\resizebox{3.3in}{!}{%
\begin{tabular}{|c|c|c|c|}
\hline
{\bf Property} & {\bf Baseline} & {\bf Network RACOONS} &{\bf Overhead} \\  \hline
Bandwidth Achieved  	  & 19.98 $\pm .04$ Mb/sec                  & 19.55 $\pm .07$ Mb/sec                                  & 2.21\%                                  \\ \hline
Average Latency  		& 6.89 $\pm .31$ ms                             & 7.11 $\pm .28$ ms                                              & 3.16\%                                                           \\ \hline
\end{tabular}
}
\label{tab:throughput}
\end{table}
\vspace{0.1 in}

 %& \cellcolor[HTML]{EFEFEF}{\color[HTML]{333333} \begin{tabular}[c]{@{}c@{}}Property\end{tabular}} & \cellcolor[HTML]{EFEFEF}\begin{tabular}[c]{@{}c@{}}Baseline\end{tabular} & \cellcolor[HTML]{EFEFEF}\begin{tabular}[c]{@{}c@{}}Network RACOONS\end{tabular} & \cellcolor[HTML]{EFEFEF}\begin{tabular}[c]{@{}c@{}}Overhead\end{tabular}

We evaluated \textit{Network RACOONS} by performing various experiments
aimed at answering the following research questions. 

\begin{enumerate}	
  \item \textbf{RQ1: Network Overhead} What is the overhead on the
  bandwidth and the latency introduced on the network by the detection
  app?
  \item \textbf{RQ2: Accuracy} What is the accuracy of detecting attacks?
  \item \textbf{RQ3: Detection Time} How long does it take for the
  detection app to detect a reconnaissance attempt after a fake packet\_in
  is received?
\end{enumerate}

We setup a small scale SDN environment to run our experiments. Our
testing environment is built with 1 virtual switch (using OpenVSwitch),
1 HP 2920-24G SDN enabled switch, and 5 virtual machines (Ubuntu 16.04) 
running on 2 physical hosts (Ubuntu 16.04) using VirtualBox. The SDN 
controller was run on a Macbook Pro running OSX. Benign traffic was
generated using 3 BeagleBone Blacks running Ubuntu, acting as low 
resource hosts. One of the three BeagleBone Blacks was used in the
simulation of a compromised switch and created the probe packets that
were changed by the malicious SDN app. The remaining two BeagleBone
Blacks  used a combination of D-ITG and iperf to generate traffic during
testing.

\begin{table}[t]
\caption{Accuracy of Network RACOONS (Row 4 with induced corner cases)}
\vspace{0.1 in}
\centering
\resizebox{3.3in}{!}{%
\begin{tabular}{|c|c|c|c|}
\hline
{\bf Packet\_ins Inspected} & {\bf Probes Detected} & {\bf False Positives} &{\bf False Negatives} \\  \hline
1133	  		&  6                & 0                                 & 0                                  \\ \hline
1744  		& 11               & 0                                 & 0                                   \\ \hline
2137			& 17		     & 0				  & 0					\\ \hline
1108 		&13		     & 2				 & 0					\\ \hline
\end{tabular}
}
\label{tab:accuracy}
\end{table}
\vspace{0.1 in}

%     & \cellcolor[HTML]{EFEFEF}{\color[HTML]{333333} \begin{tabular}[c]{@{}c@{}}Total Packet\_ins \\inspected\end{tabular}} & \cellcolor[HTML]{EFEFEF}\begin{tabular}[c]{@{}c@{}}probe packet\_ins \\ detected\end{tabular} & \cellcolor[HTML]{EFEFEF}\begin{tabular}[c]{@{}c@{}}False Positives\end{tabular} & \cellcolor[HTML]{EFEFEF}\begin{tabular}[c]{@{}c@{}}false Negatives\end{tabular} \\ \hline

\begin{table}[t]
\caption{Detection Time}
\vspace{0.1 in}
\centering
\resizebox{3in}{!}{%
\begin{tabular}{|c|c|}
\hline
{\bf Flow\_statistics Required} & {\bf Average Detection Time}  \\  \hline
No  		&  1.07 $\pm .11$ milliseconds               \\ \hline
Yes 		& 72.3 $\pm 9.2$ milliseconds                 \\ \hline
\end{tabular}
}
\label{tab:detection-time}
\end{table}
\vspace{.1 in}

%     & \cellcolor[HTML]{EFEFEF}{\color[HTML]{333333} \begin{tabular}[c]{@{}c@{}}Total Packet\_ins \\inspected\end{tabular}} & \cellcolor[HTML]{EFEFEF}\begin{tabular}[c]{@{}c@{}}probe packet\_ins \\ detected\end{tabular} & \cellcolor[HTML]{EFEFEF}\begin{tabular}[c]{@{}c@{}}False Positives\end{tabular} & \cellcolor[HTML]{EFEFEF}\begin{tabular}[c]{@{}c@{}}false Negatives\end{tabular} \\ \hline


\myparagraph{Network Overhead} We used iperf to measure the bandwidth
achieved with and without the detection app running in the presence of
other benign traffic. Since the malicious app is not a part of the
detection mechanism but is required to simulate fake packet\_in messages,
we ran the baseline with the malicious app running along with a 
forwarding app. The baseline bandwidth achieved was then compared to the
bandwidth achieved with both the malicious and detection apps running. To
measure the changes in network latency, ping was used between hosts
while other benign traffic ran in the background. The overhead introduced
by the detection app was around 2.2\% in the achievable bandwidth and 
around 3\% in the latency of the network as seen in 
Table~\ref{tab:throughput}.

\myparagraph{Accuracy} We evaluated Network RACOONS for accuracy,
specifically for false positives and false negatives. To test the
accuracy, the same environment described above was used and benign 
traffic between the hosts was generated in the same way. While benign 
traffic was being generated, spoofed packets were created at the
malicious host and the resulting packet\_in messages were further
modified by the malicious app. The number of packet\_in messages 
processed by the detection app and the number of packet\_in messages 
flagged as network reconnaissance packets were recorded for the duration
of the experiments. 

Table~\ref{tab:accuracy} shows the results of the accuracy testing. Rows
1 to 3 were run under normal configuration of the detection app, row 4
is discussed below. For each test of the normal configuration the
detection app successfully detected spoofed packet\_in messages with no
false positives or false negatives. The duration of each test run was
longer than the last, allowing the app to capture and inspect as close
to 1,000, 1,500, and 2,000 packet\_in messages as possible.

The experiment run in Row 4 of Table~\ref{tab:accuracy} simulates race
conditions that may cause false positives. These conditions are very
unlikely and did not appear during the testing of \name. To ensure the
conditions are met to cause a false positive, \name was modified to send
flow entries to switches with hard timeouts of less than .01 ms. Under
this condition, when the next switch sends a packet\_in message and the 
previous switch is queried, the corresponding flow will have expired from
the switch. Using this delay, two situations were simulated. First a 
situation where a switch is overwhelmed and takes a longer time than the 
prune interval of the ephemeral flood map to generate a packet\_in message
for a packet flooded by a previous switch. Second it simulated the
situation where a packet matches a flow, right before the flow expires
and when the app checks for the installed flow it does not find it. Both 
of these simulated conditions lead to false positives. The results of 
testing with these induced corner cases can be seen in row 4 of 
Table~\ref{tab:accuracy}. 

\myparagraph{Detection Time} Because our solution does not proactively
block the reconnaissance or probe packets, it is important to know how
much time it takes to detect such a packet and assess how much
information may have been learned by the attacker on a compromised
switch before being detected. In order to measure the time, we record
the time as soon a packet\_in message is received by the detection app
and subtract this time from the time when a message is classified as
benign or malicious. 

Table~\ref{tab:detection-time} shows the average detection time when 
the previous switch does not need to be queried is around 1 millisecond
and in the case where the detection app has to request flow statistics 
from the previous switch, the average detection time is around 72.3 
milliseconds. This shows the adversary has limited time to send a
reconnaissance probe, analyze the result and launch an attack before
their compromised switch is flagged. 
