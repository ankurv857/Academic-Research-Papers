\begin{abstract}
% Area 
This work examines previously unseen methods of active network 
reconnaissance made possible due to the advent and popularization
of Software-Defined Networking (SDN).
% Problem
The architecture of SDN allows an adversary, controlling a compromised
switch inside the network, to perform stealthy active network
reconnaissance and gather intelligence about the network. This includes 
trying to identify sensitive resources, learn about the services, identify 
network policies, and discover potential vulnerabilities that can be exploited. 
This information can then be used in forming an attack plan to exfiltrate
sensitive data.
% Solution
We propose \name, a security application running on the controller, to
detect active reconnaissance attacks.
% Methodology
\name is implemented in Python and runs on top of a Pox controller in a
software defined network, it was tested for its ability to quickly and
accurately detect network reconnaissance in a small simulated
environment.
% Results
Our testing shows \name effectively detects network reconnaissance
with no observed false negatives or false positives and an overhead of less than 3\% on
network throughput and latency.
% Take away
This work demonstrates the need to proactivley identify new network
threats introduced by SDN before it gains a more widespread deployment.
\end{abstract}
