\section{Conclusion}
\label{sec:conclusion}
We have presented our study of the consumer purchase behaviour in the context of large scale e-retail. 
We have shown that careful feature
engineering when used in conjunction with Deep Neural Networks, can be used
to predict the next (multi-timestep) logical purchase of consumers with reasonably good accuracies.
While creating our models and features we have been cognizant of
the fact that many features might not be available as is when predictions
are being generated for future period. Hence we have saved model weights and used innovative transformations 
so that we don’t necessarily have to remember complete data during forecast time,
thereby reducing the computation and memory requirements during forecast generation.

As per our initial expectations, Deep Neural Network models outperformed the ML models like Xgboost and RandomForest.
Sequence to Sequence architectures seemed to be theoretically sound choice for tackling our problem, and our results and 
observations were inline with this thought process. Model generalization and robustness
was attained using stacked generalization. As per our expectation we realized gain in accuracy post both stacking
and F\textsubscript{1}-Maximization.

At the same time we understand that computation strategy is key aspect in modelling
Millions of consumers, and we intend to explore further over this aspect by building 
Transfer Learning framework \cite{yosinski2014transferable}. We are also working to 
further improve our Sequence to Sequence neural network architectures to improve accuracy
and decrease computation time.
 

