\begin{abstract}
Modelling consumer behaviour is one of the most challenging problems for large retail firms, given the operational
scale both in terms of consumer and inventory. Retailers spend a lot of money and resources to ensure a very smooth and delightful 
consumer shopping experience by recommending and maintaining right inventory at right time. Accuractely determining 
the products consumer would purchase in the near future not only enriches the consumer shopping experience but also plays a
pivotal role in managing inventory by reducing the chances of Out of stock as well as Excess Inventory. 
This problem has been there since a long time and has been adrressed by ML researchers in conventional manner
through recommender systems and varying ML approaches. But, to my knowledge none of the models have generalized well 
in predicting the items consumer is likely to purchase at given time point due to immense non-linearity existing
in the consumer purchase pattern. In order to address this problem I present my study of consumer purchase 
behaviour using e-commerce retail data. Considering each consumer-product as an individual time series, I then 
build generalised models to predict the propensity of an item to be purchased by a consumer for a given time frame.
I demonstrate the robust performance by experimenting with different neural architectures including Multi-Layered Perceptron (MLP), 
Long Short Term Memeory (LSTM), Convolution Neural Networks (CNN), CNN-LSTM and their combined performance as meta models.
\end{abstract}
