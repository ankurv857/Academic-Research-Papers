\section{Introduction}
Consumer behaviour insights have always been one of the key business drivers for retail, specially given
fast changing consumer needs. Existing trend, competitor pricing, item reviews and marketing are some of the 
key factors driving todays consumer world in retail. While very little information is available
on future variablities of the above factors, what retailers have is large volumes of transactional data.
Retailers use conventional techniques to model transactional data for predicting consumer choice. 
While these help in estimating purchase pattern for loyal consumers and high selling items with reasonable accuracy, they 
don't perform well for the rest. Since multiple parameters interact non-linearly to define consumer purchase pattern,
traditional models are not sufficient to achieve high accuracy across thousands to millions of consumers.

In many of the retail brands, short term (4-6 weeks ahead) inventory planning is done on the basis of consumer 
purchase pattern. Given that every demand planner works on a narrow segment of item portfolio, there is high 
variability in choices that different planners recommend. Also, given their busy schedule, they have very less interaction
moments to discuss their views and insights over their recommendations. Hence, subtle effects like item cannibalization,
item affinity, pricing remains unaccounted correctly. Such inefficiencies lead to gap between consumer needs 
and item availablity, resulting in loss of business opportunities in terms of consumer churn, out of stock 
and excess inventory.

In this paper, we apply multiple deep learning architectures along with tree based machine learning algorithms
to predict the next logical purchase at consumer level. We showcase the performance of individual models with 
varying hyper configurations. We also show the performance of stacked generalization ensemble and F1-maximization 
which involves combining predictions from different models and fine tuning purchase probability cut-off at
consumer level respectively. The following section explains the overall methodology adopted to solve the problem.
It also lays out various algorithmic variants and neural network architectures applied to the problem.
Finally, section 3 describes the experiments and results obtained in various scenarios of modelling.
