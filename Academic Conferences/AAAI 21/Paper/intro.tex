\section{Introduction}
Consumer behaviour insights have always been one of the key business drivers for retail, specially given
fast changing consumer needs. Existing trend, competitor pricing, item reviews, sales and marketing are some of the 
key factors driving todays consumer world in retail. While very little information is available
on future variablities of the above factors, what retailers have is large volumes of transactional data.
Retailers use conventional techniques with the available data to model consumer purchase \cite{choudhury2019machine}. 
While these help in estimating purchase pattern for loyal consumers and high selling items with reasonable accuracy, they 
don't perform well for the long tail. Since multiple parameters interact non-linearly to define consumer purchase pattern,
traditional models are not sufficient to achieve high accuracy across thousands to millions of consumers.

In many of the retail/e-retail brands, short term (2-4 weeks ahead) inventory planning is done on the basis of consumer 
purchase pattern. Also, certain sales and marketing strategies like Offer Personalization, personalized item
recommendations are made leveraging results of consumer purchase predictions for the near future.
Given that every demand planner works on a narrow segment of item portfolio, there is high 
variability in choices that different planners recommend. Also, given their busy schedule, they have very less interaction
moments to discuss their views and insights over their recommendations. Hence, subtle effects like cannibalization
\cite{shah2007retailer}, and item-affinity remains unaccounted. Such inefficiencies lead to gap between consumer needs 
and item availablity, resulting in the loss of business opportunities in terms of consumer churn, out-of-stock, 
and excess inventory.

In this paper, we apply multiple deep learning architectures along with tree based machine learning algorithms
to predict the next logical item purchase at consumer level. We showcase the performance of individual models with 
varying hyper-parameter configurations along with the results of stacked generalization ensemble \cite{wolpert1992stacked} 
(algorithmic combination of predictions from different models) and F\textsubscript{1}-maximization (optimal purchase 
probability cut-off at consumer level).
In the next section \ref{sec:relatedwork}, we briefly discuss research work related to the problem in hand. 
Section \ref{sec:methodology} explains the overall methodology adopted to solve the problem.
It lays out and neural network architectures and various algorithmic variants applied to the problem.
Section \ref{sec:eval} describes the experiments performed, and results obtained in different modelling setups.
