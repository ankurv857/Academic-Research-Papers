\section{Introduction}
Consumer behaviour insights have always been one of the key business drivers for retail, given
fast changing consumer needs. Existing trend, competitor pricing, item reviews, sales and marketing are some of the 
key factors driving today's consumer world in retail. While very little information is available
on future variablities of the above factors, retailers do have large volumes of historical transactional data. Past study 
\cite{choudhury2019machine} has shown that retailers use conventional techniques with available data to model consumer purchase. 
While these help in estimating purchase pattern for loyal consumers and high selling items with reasonable accuracy, they 
don't perform well for the long tail. Since multiple parameters interact non-linearly to define consumer purchase pattern,
traditional models are not sufficient to achieve high accuracy across thousands to millions of consumers.

Most retail/e-retail brands, plan their short term inventory (2-4 weeks ahead)  based on consumer 
purchase pattern. Also, certain sales and marketing strategies like Offer Personalization and personalized item
recommendations are made leveraging results of consumer purchase predictions for the near future.
Given that every demand planner works on a narrow segment of item portfolio, there is a high 
variability in choices that different planners recommend. Additionally, the demand planners might not get enough opportunities 
to discuss their views and insights over their recommendations. Hence, subtle effects like cannibalization
\cite{shah2007retailer}, and item-affinity remain unaccounted for. Such inefficiencies lead to a gap between consumer needs 
and item availablity, resulting in the loss of business opportunities in terms of consumer churn, and out-of-stock
and excess inventory.

Our paper makes the following contributions -
\begin{itemize}
\item We study and present the usefulness of applying various deep learning architectures along with tree based machine 
learning algorithms to predict the next logical item purchase at consumer level.
\item We present the performance of individual models with varying hyperparameter configurations.
\item We implement stacked generalization framework \cite{wolpert1992stacked} as an ensemble method where a new model learns 
to combine the predictions from multiple existing models.
\item We design and implement F\textsubscript{1}-maximization algorithm which optimises for purchase probability cut-off 
at consumer level.
\end{itemize}
