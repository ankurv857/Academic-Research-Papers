\section{Introduction}
In most retail/e-retail settings, promotions play an important role in boosting number of transactions and profit.
Also, promotions are used on a daily basis in most of the retail environments including online retailers, 
supermarkets, fashion retailers, etc. A typical retail firm sells thousands of items and needs to design offer 
for all items at a given time. The depth of offer design is of very high importance as optimized 
offer roll out can significantly enhance the business’ bottom line.
In today’s economy, retailers offer hundreds or even thousands promotions simultaneously. Promotions aim to increase 
sales and traffic, enhance awareness when introducing new items, clear leftover inventory, bolster customer loyalty, 
and improve competitiveness.

Surprisingly, many retailers still employ a manual process based on intuition and past experience to decide the depth and 
timing of promotions. The unprecedented volume of data that is now available to retailers presents an opportunity to develop 
decision support tools that can help retailers improve promotion decisions. The promotion planning process typically 
involves a large number of decision variables, and needs to ensure that the relevant business constraints (called 
promotion business rules) are satisfied.