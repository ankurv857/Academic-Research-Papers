\section{Introduction}
In most retail settings, promotions play an important role in boosting the sales and traffic of the organisation.
Promotions aim to enhance awareness when introducing new items, clear leftover inventory, bolster customer loyalty, 
and improve competitiveness. Also, promotions are used on a daily basis in most retail environments including online retailers, 
supermarkets, fashion retailers, etc. A typical retail firm sells thousands of items in a week and needs to design offer 
for all items for the given time period. Offer design decisions are of primary importance for most retail firms, 
as optimal offer roll out can significantly enhance the business’ bottom line.

Most retailers still employ a manual process based on intuition and past experience of the category managers
to decide the depth and timing of promotions. The category manager has to manually solve the promotion optimization problem
at consumer-item granularity, i.e., how to select an optimal offer for each period in a finite horizon so as to maximize the 
retailer’s profit. It is a difficult problem to solve, given that promotion planning process typically 
involves a large number of decision variables, and needs to ensure that the relevant business constraints or offer rules
are satisfied. The high volume of data that is now available to retailers presents an opportunity to develop 
machine learning based solutions that can help the category managers improve promotion decisions.

In this paper, we propose deep learning with multi-obective optimization based approach to solve 
promotion optimization problem that can help retailers decide the promotions for multiple items while accounting 
for many important modelling aspects observed in retail data. The ultimate goal here is to maximize net revenue and
consumer retention rate by promoting the right items at the right time using the right offer discounts at 
consumer-item level. Our contributions in this paper include a) Temporal Convolutional Neural Network architecture
with hyperparameter configurations to predict the item purchase probability at consumer level for the given time period. 
b) Design and implementation of F\textsubscript{1}-maximization algorithm which optimises for purchase 
probability cut-off at consumer level. c) Methodology to estimate offer elasticity of purchase probability at consumer 
item granularity. d) Constraint based multi-obective optimization technique to estimate optimal offers 
at consumer-item granularity.