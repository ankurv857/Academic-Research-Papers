\section{Methodology}
\label{sec:methodology}
\subsection{Data}
Walmart US Stores POS (point of sales) / transaction data exists at the most granular level of sales data of an item at 
a single transaction. Methodology leverages transaction data which is transformed to a UPC – Week level. Filtering and 
processing of data includes exclusion of sales occurring under the bracket of clearance and markdown. Rationale for 
this is to exclude data points having sales at prices which are far off from the base price of a given UPC since the 
model building for impact of price is accurate when changed in some delta range of the base price which comprises major 
proportion of the training data set. Unit sales are rolled up at a week level for a given UPC. Price point occurring 
in majority of stores is assigned as the price of a given UPC for that week. One price point is identified at each week 
since Walmart pricing landscape is moving towards national level pricing which implies existence of a single price across 
all store at a given point of time. The solution is constructed keeping in view national pricing landscape.

Cross price effects are an important component of the demand equation to establish causal relationship between unit 
sales of parent item and price of the halo and substitute UPCs. Top three substitute UPCs within the category of a parent 
item are identified and the price points at week level is integrated with the base transaction data of the parent UPC at 
the week level. Top three affined UPCs of a parent UPC outside the category and within the department of parent UPC are 
identified and the price points at week level is integrated with the base transaction data of the parent UPC at the week level.

As part of base transaction data, month flags are introduced as binary variables indicating whether it belongs to one of 
the twelve months of the year or not. National holidays are also incorporated as binary flags for the following events 
- Mother’s Day, Thanksgiving Day, Valentine’s day, Ash Wednesday, Father’s Day, St. Patrick’s Day, Good Friday, Independence Day,
 Easter, New year, Kwanzaa, Labor Day, Memorial Day, Christmas, Halloween, Cyber Monday.

Promotion indicator is created at a week UPC level as a continuous variable which is calculated as percentage of stores 
on promotion including rollbacks, store tab and competition tab.

\subsection{Demand Equation}
First exploratory exercise undertaken was to build most granular level models at the UPC level which leveraged data for 
each UPC for constructing a UPC level demand equation. However, in Walmart universe there is a less proportion of UPCs with 
significant historical price changes. Due to this constraint, only a few UPCs out of the extensive Walmart universe had a 
model with price as a significant feature. To tackle this issue, multiproduct approach was tested and implemented. 
Multiproduct approach is proposed for building models which specifies the demand equation having unit sales per store 
of a UPC as a target variable and explanatory variables mentioned in section III(A). Multiproduct approach refers to 
building line, fine-line and category level models where data for all UPCs belonging to a line, fine-line and category 
respectively are clubbed together for modelling. This approach was identified and proposed to overcome challenges of 
limited data points for UPCs with less transaction history, to identify potential causal price impacts for UPCs which 
don’t have historical price changes and potentially learn from the behavior of UPCs falling within the group 
(line, fine-line and category) of the UPC in consideration. Modelling techniques used is specified in Section III (B).

\subsection{Elasticity Estimates using Simulation}
Self-Price elasticity estimate for a UPC is one of the crucial components that defines the characteristic of UPC and enables 
in computing sales lift and therefore to identify potential UPCs for investment.

Elasticity estimate is arrived at using a simulation procedure. For each UPC, base price is identified and 6 price variants 
are generated from the base price. They include +5%, +10%, +15%, -5%, -10%, -15% from the base price of a UPC. Unit sales is 
scored at these 7 price points using the multiproduct model’s demand equation as mentioned in section II(B). Percentage change 
in unit sales is calculated between successive price points and is divided by percentage price change to arrive at an 
elasticity estimate. Average of elasticity estimates at simulated price points is used to arrive at an elasticity figure 
for a given UPC.

\subsection{Model Explanatory Variables}
Integrated data described in the earlier sections consists of a set of variables such as self-price, price of halo UPCs, 
substitute prices, national events flags, monthly flags, and promotion indicator. Different transformations of these 
variables are used as features. Features used for building line and fine-line level models are stated in Table I.

Most of the features are expressed as relative variables since UPCs with varying magnitude of self, halo and substitute 
prices are used for model building. This has been incorporated to ensure standardization of variables. An offset is also 
introduced into the model formulation which has been discussed in detail in section III (B).

Several interactions are also incorporated as a part of the feature set such as interaction of promotion indicator 
with relative self-price to identify any statistically significant additional impact of price change if it co-occurs 
with a promotion activity. Binary flags for certain UPCs which contribute to 90% of cumulative sales in the multiproduct 
group has also been incorporated to identify any additional impact if the UPC in consideration belongs to higher sales 
contribution UPCs. These high contribution UPC features are also interacted with relative self-price to identify any 
statistically significant additional impact of price change for a high sales contribution UPC.

Features for category level models have been stated in Table II. Certain week and year related variables have been introduced 
to capture seasonality effects. Week of transaction for a UPC has been introduced to capture the effects accruing to maturity 
of transaction cycle for a UPC.

Mean volume of UPC has been introduced for the purpose of target encoding aimed at providing a baseline to the target 
variable. Volume bucket has been introduced as an interaction between mean price and mean volume of a UPC. Holiday flag 
has been incorporated to capture any additional impact due to existence of a national holiday period. Promotion indicator 
has been introduced to capture the effects on unit sales due to varying promotion intensity. Price change percentage 
variable has been introduced for capturing the impact on target variable due to deviation in price from the base price of a UPC.

\subsection{Modelling Approach}
Multiproduct models at a line and fine line level have been built using regularized elastic net regression 
technique with unit sales per store at a weekly being the target variables and the list of features stated in Table I. 
Regularized regression technique has been used to retain the explainable nature of solution, avoid overfitting and to 
identify the potential causal impact of changes in the features. Offset for the target variable has been used as rolling 
average unit sales per store of a UPC. The family of distribution for dependent variable has been set as “Poisson” to 
enable multiplicative structure and specification of the demand equation. The time period for rolling average has been 
fixed at 4,13 and 52 weeks and based on the availability of data for maximum lag period, offset has been assigned for 
each UPC – week level record. R package “glmnet” [1] has been used for this computation. 

All self-price, halo and substitute features are constrained to be non-positive. The rationale behind this is to 
ensure an inverse relationship between self-price and unit sales. Rationale for constraining coefficient of relative 
substitute price feature is to ensure a direct relationship between substitute prices and unit sales. Similarly halo 
relative price is constrained to have a negative coefficient to establish an inverse relationship between price of an 
affined UPC and unit sales of parent UPC.  Promotion indicator is constrained to have a non-negative coefficient to 
establish a direct relationship between unit sales and intensity of promotion activity. 

Modelling data is divided into two parts comprising of approximately 104 weeks of data for each UPC as the training 
set and around 13 weeks of data for each UPC as testing data for out of time testing using MAPE (mean absolute percent error) 
as an error metric.  Model coefficients are arrived at using 10-fold cross validation on the training set.

Category level models have been built using linear boosting techniques [2] based on parallel coordinate descent. 
Boosting technique has been implemented to leverage larger pool of available data points. Target variable is unit 
sales per store of a UPC. The set of features have been mentioned in Table II and rationale for incorporating the 
features has been explained in detail in section III(B). The model is specified to have an additive structure.
Hyperparameter tuning has been implemented to arrive at a learning rate of 0.5. Number of iterations has been specified as 
1500. MAPE (mean average percentage error) has been used as the evaluation metric for training the model.
