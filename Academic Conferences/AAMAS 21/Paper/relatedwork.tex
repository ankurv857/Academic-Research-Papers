\section{Related Work}
\label{sec:relatedwork}
In the past few years, various machine learning methods for predicting consumer purchase pattern have been analyzed in
the academia field and few of them are often used by ML practitioners. In most cases those approaches are based on extracting 
consumer's latent characteristics from its past purchase behavior and applying statistical and  
ML based formulations \cite{fader2009probability, choudhury2019machine}. 
Some of previous studies have analyzed the use of random forest and Xgboost techniques in order to predict 
consumer retention, where past consumer behavior was used as potential explanatory variable 
for modelling such patterns. In one such study \cite{martinez2020machine}, the authors develop a model for predicting wether a 
consumer performs a purchase in prescribed future time frame based on historical purchase information such as the number
of transactions, time of the last transaction, and the relative change in total spending of a consumer. 
They found gradient boosting to perform best over test data. We propose neural network architectures with entity embeddings
\cite{guo2016entity} which outperforms the gradient boosting type of models like Xgboost \cite{chen2016xgboost}. 

Over Time Series part of our problem,
close to our work is Deep Neural Network Ensembles for Time Series Classification \cite{fawaz2019deep}. 
In this paper, authors show how an ensemble of multiple Convolutional Neural Networks can improve upon the 
state-of-the-art performance of individual neural networks. They use 6 deep learning classifiers 
including Multi Layer Perceptron, Fully Convolutional Neural Network, Residual Network, 
Encoder \cite{serra2018towards}, Multi-Channels Deep Convolutional Neural Networks \cite{zheng2014time} and 
Time Convolutional Neural Network \cite{zhao2017convolutional}. The first three were originally proposed in \cite{wang2017time}. 
We apply the concept of entity embeddings \cite{guo2016entity} along with neural network architectures
like Multi Layer Perceptron, Long Short Term Memory (LSTM), Temporal Convolutional Networks (TCN) \cite{lea2016temporal} and 
TCN-LSTM \cite{karim2017lstm}, and prove their application in modelling consumer-item time series data.
