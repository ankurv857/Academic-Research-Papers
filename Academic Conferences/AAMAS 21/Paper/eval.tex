\section{Experiments and Results}
\label{sec:eval}

\subsection{Loss Function}
The models have been judged on various criteria for their Mean Absolute Percent Error.
As discussed in section III(B), 13 weeks of data for each UPC has been kept as test data. We calculate MAPE for each of
these UPCs by doing prediction on test data using the trained model. It is an indicator of how accurate the models are. 
A model has been considered a legitimate model only if the MAPE is less than 30%.

\subsection{Model Stability} 
The products available in Walmart have dynamic nature, their nature and characteristics keep on changing over time. 
In this regard, it becomes very important to ensure that models which have been built are robust and successfully 
capture the dynamic nature of the products. 

To do this, a model for an UPC has been trained on datasets having same features but are of different time periods. 
This has been done on five different datasets for an UPC and elasticity values have been generated using the simulation 
technique as described in section II(C). The average change in coefficient values of price variables as well as elasticity 
values for all the UPCs was less than 10% across all of these time periods. This test makes sure that the model is not 
getting overfit on the train data. 

\subsection{Volume Lift Accuracy}  
As discussed in previous sections, most of the products in the Walmart eco-system do not go through significant 
number of price changes, in order to estimate the performance of the models during a price change, volume lift 
accuracy has been calculated. 

This has been done by considering all those weeks where     
price change has occurred in train and test data. In all those weeks, next eight weeks of actual 
as well as predicted volume have been taken and MAPE, discussed above, has been calculated. The rationale behind
taking next eight weeks is that the effect of a price change such as jump in demand has always been observed at 
least a week after the price change. One of the reasons behind this lag is customers take time to internalize the price 
change and react to it.

\subsection{Discussion}  
Walmart landscape. 

Table III discusses the overall coverage of our models for 40 odd categories available in Walmart eco-system. 
To get coverage of a model, we only consider UPCs which have models with MAPE less than 30%. In addition to this, 
only the UPCs which have elasticity in range [-4,0) have been considered. The baseline for the coverage calculation 
includes only those items which have been sold in more than 2000 stores in last 1 year to include only national items not 
regional items. 

Total number of UPCs column in Table III gives the baseline for the categories and other columns depict the coverage 
from three different models. The last column is the overall coverage which has been found by combining the results of 
three different models.

Consider the category number 1867 in Table III, this category has 729 UPCs which have been sold in more than 2000 stores 
in 1 year, out of which ,561 UPCs have been successfully captured by these three models i.e. 561 UPCs are more than 70% 
accurate and have elasticity in [-4,0). Similarly, category 1483 has exceptionally good coverage of 97%. However, coverage 
is not satisfactory for all the categories for e.g., category 8539(Television) has overall coverage quite low. One of the 
reasons might be highly elastic nature of electronic appliances. This shows that other approaches also need to be tried out
 to capture these nuances. 

Table IV summarizes the coverage from three models. Overall, the line, fineline and category models are capturing 70% 
of national items for these 40 odd categories. 

